\section{Experiments}
\label{sec:experiments}

We  utilize a JHU APL developed Renderer and Camera Emulator (RCE) to simulate a range of imagery.  We will also use NEOWISE imagery for algorithm testing.  

\subsection{Simulated Imagery}
\label{ssec:simulated}

Asteroids are  modeled as spherical blackbody-like emitters (emissivity is less than 1), with a cross-sectional area that approximates the sizes of actual asteroids and surface temperatures typical of sun-illuminated asteroids in an Earth-like orbit.  Similar to the way stars (see below) are modeled, the radiation emitted is modeled using a form of Planck?s equation:
 
 The asteroids are assumed to have a nominal temperature of 200 K due to solar heating and emissivities in the range from 0.9 to 0.98. Therefore, their spectral radiance would look like what is shown in Figure 3.  For this 200 K blackbody, the peak in the radiance occurs at a wavelength of 14.5 µm.
 
 The RCE uses stellar data available as part of the Two Micron All Sky Survey (2MASS), a stellar survey that scanned the entire sky in three IR bands (centered at 1.25 µm, 1.65 µm, and 2.17 µm, respectively).  The 2MASS catalog also incorporates data in two visible bands from other surveys. 
 
Results for each step of the pipeline are shown:
 Image Registration in Figure~\ref{}. Logical differencing in Figure~\ref{}. (See Figure ~\ref{} for trajectory detection on an example image triplet).    As is shown in Figure~\ref{}, using trajectory verification on a greater number of images in the sequence allows us to quickly disambiguate and reject false trajectories (in this case this trend is readily apparent when going from a triplet to a quadruplet of images).

An example of ground truth trajectory derived from the simulated MWIR imagery is shown in Figure~\ref{}. In this figure, the simulated images are super-imposed in order to visualize the asteroid trajectory in a single image. Figure~\ref{} shows the final trajectories detected for one triplet and one quadruplet of the simulated MWIR dataset.
 
\subsection{Real Imagery}
\label{ssec:real}

\subsubsection{NEAT}
\subsubsection{CATALINA}


\subsection{performance characterization}
We use the following metrics for algorithm testing and validation.  These are used to characterize the performance of the algorithm for the three successive stages: the initial detection of objects, the detection of moving objects and the detection of trajectories:
1.	Precision: The fraction of retrieved/detected objects (celestial bodies, moving objects, trajectories) that are relevant, i.e. correspond to correct detections.
2.	Recall: The fraction of relevant (true) objects (celestial bodies, moving objects, trajectories) that are actually detected/retrieved.
3.	Receiver Operating Characteristic (ROC): A plot of the probability of detection as a function of the probability of false alarm generated by varying a significant parameter of the algorithm (such as the initial detection threshold).  This will detail the performance of the system as the parameters of the image processing pipeline are varied. 
4.	Area under the ROC curve (AUC) that characterizes the performance of detection vs. false alarm with a single metric.
5.	Localization error: For those correctly detected asteroids, we will characterize the localization error by computing the 3D (angular pointing error).  The simulation and the test image sets have the ground truth location of the asteroid in image coordinates.  This true location is compared with the detected trajectory to determine localization error.

The above measures are computed and aggregated over a corpus of images corresponding to a simulated platform and camera.
\subsection{Footprint characterization}
include here some information to the effect that this algorithm is deployable on a reference platform.
Efforts are currently being conducted to translate this algorithm in ...