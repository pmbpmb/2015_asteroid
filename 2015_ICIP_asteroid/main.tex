% Template for ICIP-2015 paper; to be used with:
%          spconf.sty  - ICASSP/ICIP LaTeX style file, and
%          IEEEbib.bst - IEEE bibliography style file.
% --------------------------------------------------------------------------
\documentclass{article}
\usepackage{spconf,amsmath,graphicx}
\usepackage{tikz}
\usetikzlibrary{shapes.geometric, arrows}
% Example definitions.
% --------------------
\def\x{{\mathbf x}}
\def\L{{\cal L}}

% Title.
% ------
\title{V151231: NEAR EARTH BODY DETECTION and linking}
%
% Single address.
% ---------------
\name{ us the authors us the authors us the authors 
\thanks{This work is supported by NASA Grant XYZ.}}
\address{The Johns Hopkins University  \\ $^{1}$Dept. of Computer Science, $^{2}$Dept. of Applied Mathematics and Statistics, $^{3}$Applied Physics Laboratory}
%
% For example:
% ------------
%\address{School\\
%	Department\\
%	Address}
%
% Two addresses (uncomment and modify for two-address case).
% ----------------------------------------------------------
%\twoauthors
%  {A. Author-one, B. Author-two\sthanks{Thanks to XYZ agency for funding.}}
%	{School A-B\\
%	Department A-B\\
%	Address A-B}
%  {C. Author-three, D. Author-four\sthanks{The fourth author performed the work
%	while at ...}}
%	{School C-D\\
%	Department C-D\\
%	Address C-D}
%
\begin{document}
%\ninept
%
\maketitle
%
\begin{abstract}
Most asteroid population discovery has been accomplished to-date by earth-based telescopes. It is speculated that most of the smaller Near Earth Objects (NEOs), up to 140 meters in diameter, whose impact can create substantial city-size damage, have not yet been discovered.  Many asteroids  cannot be detected with an earth-based telescope given their size or the location of the Sun.  Our objective is therefore to develop an efficient asteroid detection and linking algorithm that can be hosted on-board a spacecraft.  By having on-board algorithms, the system would also minimize the need to downlink entire images taken by a space-based telescope. We describe one such image processing pipeline we have developed for onboard asteroid detection and characterize its performance.
 
\end{abstract}
%
\begin{keywords}
Asteroids detection, identification. 
\end{keywords}
%
\section{Introduction}
\label{sec:intro}

NASA has a congressional mandate to discover all Near-Earth Objects (NEOs) at least 1 kilometer in diameter.  Fortunately, 95\% of the NEOs larger than 1 km that have been discovered are likely not to impact Earth.  
Near-Earth Object search programs~\cite{stokes2002near} are currently almost exclusively accomplished by earth-based telescopes such as MIT's LINEAR \cite{evans2003detection}  project, the NEAT~\cite{neat2014} program, the Catalina Sky Survey, or  Pan-STARRS~\cite{denneau2013pan}.  An exception is JPL's spacecraft-based WISE telescope brought out of hibernation to characterize NEOs in the 3.4 and 4.6 micron infrared bands~\cite{wise2014} (called NEOWISE). 

It is notable, however, that the NEO that has impacted Chelyabinsk, Russia on 15 February 2013 was only about 17 meters in size.  The impact of a 50 meter asteroid that caused the Tunguska Event of 1908 could have destroyed an entire city or metropolitan area. It is estimated that only a relatively small fraction of those so called ``city-killing'' asteroids, particularly objects less than 140 meters in diameter, have been discovered to date. Because of their size, atmospheric effects and the location of the sun, some of these NEOs cannot easily be detected with an earth-based telescope.  

Our focus here is therefore on developing an algorithm that can be hosted on-board a spacecraft.  Understanding that there are processing and resources (memory) constraints on-board, our algorithm design needs to not only meet the performance objectives of detecting and identifying asteroids using a space-based telescope, but it also needs to have a small footprint to be implementable with the limited on-board resources.  This paper describes an agile algorithm candidate that is being investigated as well as our use of representative real and simulation imagery for testing it.

Prior image analysis work for asteroid detection for ground-based image observations is summarized: In~\cite{denneau2013pan}, a reference processing system for detection and identification of asteroids -- named the Pan-STARRS Moving Object Processing System (MOPS) --  is described. This pipeline aims at identifying moving objects in our solar system and linking those detections within and between night observations. It attributes those detections to known objects, calculates initial and differentially corrected orbits for linked detections, recovering detections when they exist, and orbit identification. Most proposed pipelines for earth based detection include a step to combine images to create a high S/N static-sky image that is subtracted from the current master image to obtain a difference image containing only transient sources. Examples include \cite{shao2014finding}, where a shift-and-add technique is used to improve  signal to noise ratio and then synthetically creating long exposure images to facilitate the detection of trajectories. A related shift-and-add method using a median  image rather than an average image is reported in~\cite{yanagisawa2005automatic}. A match filter is used for asteroid detection and matching in \cite{gural2005matched}. In \cite{kubica2005variable,kubica2005multiple,kubica2007efficient}, tree based searches (including KD-trees) are used for efficient linking of successive asteroids detections and finding sets of observation points that can be fitted with an inherent motion model, through an exhaustive search for all possible linkages that satisfy the expected model constraints.
	
In Section~\ref{sec:approach} we describe a small-footprint pipeline (with regard to memory and CPU usage) that can be deployed on an on-board system. This approach uses Principal Component Analysis based searches to link object detections into trajectories. Since space-based asteroids surveys such as NEOWISE are still rare, we describe in Section~\ref{sec:experiments} a validation method using space-based simulated images as well as earth-based image datasets that contain associated ground truth.

\section{Approach}
\label{sec:approach}

We describe the main components of our image processing pipeline (shown also in the flow chart in figure \ref{IPP}) addressing detection and linking of asteroids seen in images acquired from space-based platforms.  At a high level, the pipeline is summarized as follows: as input to the pipeline is a sequence of time-lapse images. Pre-processing and image registration are used to bring the images into alignment with a common image of reference.  Detection of bright bodies is done via thresholding, followed by logical differencing between each image and a reference image containing objects that are common to all registered images in the sequence of images.  After differencing, which allows us to detect moving objects, we use an additional step to filter detections based on size and shape considerations (using tools such as connected components and morphological filters).  A list of detection coordinates is then generated.  Candidate trajectories are then generated for these coordinates by efficiently checking rectilinearity by using both Principal Component Analysis (PCA) and 2d-trees.  A last verification step further checks additional conditions such as the fact that candidate trajectories must be composed of temporally consecutive mover observations.  Component module are detailed next. 

\tikzstyle{block} = [rectangle, minimum width=1.8cm, minimum height=1.5cm, text centered, text width=1.8cm, draw=black, fill=blue!15]
\tikzstyle{arrow} = [thick,->,>=stealth]
\begin{figure}[b]
\begin{tikzpicture}[node distance=2.25cm]
\node (proc) [block] {Image Pre-Processing};
\node (reg) [block, right of=proc] {Image Registration};
\node (diff) [block, right of=reg] {Logical Differencing};
%\node (det) [block, right of=diff, yshift=1.2cm] {Trajectory Detection};
\node (det) [block, right of=diff] {Trajectory Linking};
\draw [arrow] (proc) -- (reg);
\draw [arrow] (reg) -- (diff);
\draw [arrow] (diff) -- (det);
\end{tikzpicture}
\vspace{-0.7cm}
\caption{Image Processing Pipeline}
\label{IPP}
\end{figure}

{\bf Image Pre-Processing}
%We standardize the input images by using a series of photometric and geometric transformations.  
This step includes employing techniques to reduce noise (e.g. median filter) and artifacts that depend on the acquisition device.   %This coordinate information can then be stored in the FITS (Flexible Image Transport System) image header for later use in the Trajectory Verification stage.

{\bf Image Registration}
%Image registration aims at finding the transformation that would align multiple images of the same scene that are obtained at different times from slightly different viewpoints.  
We aim to align each image in the sequence to a common reference image so that the stars in the background line up in all images.  In the case of a triplet of images, typically the second image is used as the reference image.  We estimate and then apply the necessary similarity transformation (translation, rotation, scaling) and/or skew (for a full affine transformation) to all images in the sequence such that image objects (stars) are mapped into the same pixel location and so that all transformed images have the same spatial resolution.  Image Registration using mutual information~\cite{viola1997alignment} and cross-correlation as similarity measures are used. 

{\bf Image Logical Differencing}
A global thresholding is applied to the registered image for detecting asteroids and suppressing background noise.  As asteroids are typically very faint compared to the surrounding stars, the selection of the detection threshold impacts false alarm rate.  Thresholding yields binary detection images.  The set of all binary images is then used to generate an intersection image that contains objects that occur in all images in the sequence.  This is followed by logical differencing whereby we produce a set of difference images by intersecting the corresponding binary image with the negative of the common intersection image.  This operation provides a list of candidate detections for each image in the sequence.  While the logical differencing results in good detections, additional artifacts such as crater-like formations (see Fig.~\ref{IPP_NEAT_Layout1}) are seen as a result of some of the celestial bodies being over-exposed.  To mitigate this artifact, we subsequently perform filtering out of hollow objects as well as filtering based on object size.

{\bf Trajectory Linking}
The list of centroids of moving objects obtained from image differencing defines a set of candidate rectilinear trajectories.  The goal is to find a subset of centroids that fit a linear model. Even though the model is simple, the set of filtered centroids potentially has a high number of noisy points (falsely detected movers), and the cardinality of the set of all candidate trajectories increases exponentially with the number of detections, thus requiring subsequent pruning of this set. We use a combination of PCA and 2d-trees in order to find the trajectories efficiently. Unlike MOPS~\cite{denneau2013pan} and CSS, we do not set an upper limit on the velocity of the asteroid, and hence do not risk missing potential fast moving targets. Given a sequence of images, we form all the possible trajectories connecting the detections in the first and last frames. There are $O(n^2)$ trajectories, where $n$ is the number of detections per images. We then find the point of intersection of each of these trajectories with the frames in the middle. We construct a 2 dimensional tree for all the frames in the middle, and perform a range search on the tree to generate the detections that lie within a small radius of the point of intersection. This query can be done on $O(log(n))$ time on average. Once we find a collection of such points that potentially form a linear trajectory, we perform PCA and compute the ratio of eigenvalues, $\lambda_{1}/(\lambda_{1}+ \lambda_{2})$ in order to develop a line confidence score for each candidate trajectory, and choose lines for which this ratio exceeds a threshold. Using the candidate trajectories thus found, we then enforce temporal order of detection by using the sign of the projection on the principal eigenvector.  As the final step, we further eliminate false positives by ensuring that the distance between projections is proportional to the time interval between images. Compared to a brute force line search of $O(n^3)$ for a triplet of images, our algorithm takes $O(n^2log(n))$ time.(TODO: Verify this.)
%Compared to a brute force line search ($O(n^k)$, where $k$ is the number of images in the sequence) our algorithm takes $O(n^2log(n))$ time.

\section{Experiments}
%\section{Experiments}
%\label{sec:experiments}
%
%\subsection{Simulated Imagery}
%\label{ssec:simulated}
%
%\subsection{Real Imagery}
%\label{ssec:real}
%
%\subsubsection{NEAT}
%\subsubsection{CATALINA}
%
%
%\subsection{performance characterization}

\label{sec:experiments}

We  detail experiments using a JHU APL developed Renderer and Camera Emulator (RCE) to simulate a range of imagery and real imagery from the NEOWISE dataset.  

\subsection{Simulated Imagery}
\label{ssec:simulated}

Asteroids are  modeled as spherical blackbody-like emitters (emissivity is less than 1), with a cross-sectional area that approximates the sizes of actual asteroids and surface temperatures typical of sun-illuminated asteroids in an Earth-like orbit.  Similar to the way stars (see below) are modeled, the radiation emitted is modeled using a form of Planck?s equation:
 
 The asteroids are assumed to have a nominal temperature of 200 K due to solar heating and emissivities in the range from 0.9 to 0.98. Therefore, their spectral radiance would look like what is shown in Figure 3.  For this 200 K blackbody, the peak in the radiance occurs at a wavelength of 14.5 µm.
 
 The RCE uses stellar data available as part of the Two Micron All Sky Survey (2MASS), a stellar survey that scanned the entire sky in three IR bands (centered at 1.25 µm, 1.65 µm, and 2.17 µm, respectively).  The 2MASS catalog also incorporates data in two visible bands from other surveys. 
 
Results for each step of the pipeline are shown:
 Image Registration in Figure~\ref{}. Logical differencing in Figure~\ref{}. (See Figure ~\ref{} for trajectory detection on an example image triplet).    As is shown in Figure~\ref{}, using trajectory verification on a greater number of images in the sequence allows us to quickly disambiguate and reject false trajectories (in this case this trend is readily apparent when going from a triplet to a quadruplet of images).

An example of ground truth trajectory derived from the simulated MWIR imagery is shown in Figure~\ref{}. In this figure, the simulated images are super-imposed in order to visualize the asteroid trajectory in a single image. Figure~\ref{} shows the final trajectories detected for one triplet and one quadruplet of the simulated MWIR dataset.
 
\subsection{Real Imagery}
\label{ssec:real}

\subsubsection{NEAT}
The following figures show the results at all stages of the pipeline for one triplet of images of the 2002-CY46 asteroid obtained from the NEAT [24] survey.  
%\begin{figure*}
%\minipage{0.33\textwidth}
%  \includegraphics[width=\linewidth]{Figures/NEAT1.pdf}
%\endminipage\hfill
%\minipage{0.33\textwidth}
%  \includegraphics[width=\linewidth]{Figures/NEAT2.pdf}
%\endminipage\hfill
%\minipage{0.33\textwidth}
%  \includegraphics[width=\linewidth]{Figures/NEAT3.pdf}
%\endminipage
%\caption{2002 CY46 Triplet Near Earth Asteroid Tracking (NEAT) system archive}
%\label{fig:NEAT_Images}
%\end{figure*}

\newcommand{\imgWidth}{0.14\textwidth}
\begin{figure}[h]
\begin{center}$
\begin{array}{c@{\hspace{0.5em}}c@{\hspace{0.5em}}c}
\includegraphics[width=\imgWidth]{Figures/NEAT1.pdf} &
\includegraphics[width=\imgWidth]{Figures/NEAT2.pdf} &
\includegraphics[width=\imgWidth]{Figures/NEAT3.pdf} \\
\includegraphics[width=\imgWidth]{Figures/NEATImageReg12.pdf} &
\includegraphics[width=\imgWidth]{Figures/NEATImageReg32.pdf} \\
\includegraphics[width=\imgWidth]{Figures/NEATImageDiff1.pdf} &
\includegraphics[width=\imgWidth]{Figures/NEATImageDiff2.pdf} &
\includegraphics[width=\imgWidth]{Figures/NEATImageDiff3.pdf} \\
\includegraphics[width=\imgWidth]{Figures/NEATFilteredCentroids1.pdf} &
\includegraphics[width=\imgWidth]{Figures/NEATFilteredCentroids2.pdf} &
\includegraphics[width=\imgWidth]{Figures/NEATFilteredCentroids3.pdf}
\end{array}$
\end{center}
\caption{Option 1. Image Processing Pipeline results. 
First row: 2002 CY46 Triplet images taken 10 minutes apart. Near Earth Asteroid Tracking (NEAT) system archive. 
Second row: Image Registration results for the CY46 Triplet.  Left: Image-1 registered to Image-2. Right: Image-3 registered to Image-2.
Third row: Image Differencing results for the CY46 Triplet. (Artifacts such as crater-like formations are seen in the difference images above. This is the result of some celestial bodies being over-exposed.) 
Fourth row: Image Differencing results for the CY46 Triplet. Filtered centroids in each image of the sequence.)}
\end{figure}

\newcommand{\imgWidthMedium}{0.23\textwidth}
\begin{figure*}[t]
\begin{center}$
\begin{array}{c@{\hspace{.5em}}c@{\hspace{0.5em}}c@{\hspace{0.5em}}c}
\includegraphics[width=\imgWidthMedium]{Figures/NEAT1.pdf} &
\includegraphics[width=\imgWidthMedium]{Figures/NEATImageReg12.pdf} &
\includegraphics[width=\imgWidthMedium]{Figures/NEATImageDiff1.pdf} &
\includegraphics[width=\imgWidthMedium]{Figures/NEATFilteredCentroids1.pdf} \\
\includegraphics[width=\imgWidthMedium]{Figures/NEAT2.pdf} &
 &
\includegraphics[width=\imgWidthMedium]{Figures/NEATImageDiff2.pdf} &
\includegraphics[width=\imgWidthMedium]{Figures/NEATFilteredCentroids2.pdf} \\
\includegraphics[width=\imgWidthMedium]{Figures/NEAT3.pdf} &
\includegraphics[width=\imgWidthMedium]{Figures/NEATImageReg32.pdf} &
\includegraphics[width=\imgWidthMedium]{Figures/NEATImageDiff3.pdf} &
\includegraphics[width=\imgWidthMedium]{Figures/NEATFilteredCentroids3.pdf} 
\end{array}$
\end{center}
\caption{ Option 2. Image Processing Pipeline results. 
First column: 2002 CY46 Triplet images taken 10 minutes apart. Near Earth Asteroid Tracking (NEAT) system archive. 
Second Column: Image Registration results for the CY46 Triplet.  Top: Image-1 registered to Image-2. Bottom: Image-3 registered to Image-2.
Third Column: Image Differencing results for the CY46 Triplet. (Artifacts such as crater-like formations are seen in the difference images above. This is the result of some celestial bodies being over-exposed.) 
Fourth Column: Image Differencing results for the CY46 Triplet. Filtered centroids in each image of the sequence.)}
\end{figure*}

\begin{figure*}[t]
\begin{center}$
\begin{array}{ccc}
\includegraphics[width=0.33\textwidth]{Figures/NEAT1.pdf} &
\includegraphics[width=0.33\textwidth]{Figures/NEAT2.pdf} &
\includegraphics[width=0.33\textwidth]{Figures/NEAT3.pdf} \\
\includegraphics[width=0.33\textwidth]{Figures/NEATImageReg12.pdf} &
\includegraphics[width=0.33\textwidth]{Figures/NEATImageReg32.pdf} \\
\includegraphics[width=0.33\textwidth]{Figures/NEATImageDiff1.pdf} &
\includegraphics[width=0.33\textwidth]{Figures/NEATImageDiff2.pdf} &
\includegraphics[width=0.33\textwidth]{Figures/NEATImageDiff3.pdf} \\
\includegraphics[width=0.33\textwidth]{Figures/NEATFilteredCentroids1.pdf} &
\includegraphics[width=0.33\textwidth]{Figures/NEATFilteredCentroids2.pdf} &
\includegraphics[width=0.33\textwidth]{Figures/NEATFilteredCentroids3.pdf}
\end{array}$
\end{center}
\caption{Option 3. Image Processing Pipeline results. 
First column: 2002 CY46 Triplet images taken 10 minutes apart. Near Earth Asteroid Tracking (NEAT) system archive. 
Second Column: Image Registration results for the CY46 Triplet.  Top: Image-1 registered to Image-2. Bottom: Image-3 registered to Image-2.
Third Column: Image Differencing results for the CY46 Triplet. (Artifacts such as crater-like formations are seen in the difference images above. This is the result of some celestial bodies being over-exposed.) 
Fourth Column: Image Differencing results for the CY46 Triplet. Filtered centroids in each image of the sequence.)}
\end{figure*}

%\begin{figure}[b]
%\minipage{0.24\textwidth}
%  \includegraphics[width=\linewidth]{Figures/NEATImageReg12.pdf}
%\endminipage\hfill
%\minipage{0.24\textwidth}
%  \includegraphics[width=\linewidth]{Figures/NEATImageReg32.pdf}
%\endminipage\hfill
%\caption{Image Registration results for the CY46 Triplet.  Left: Image-1 registered to Image-2. Right: Image-3 registered to Image-2.}
%\label{fig:NEAT_Registration}
%\end{figure}

%\begin{figure*}
%\minipage{0.33\textwidth}
%  \includegraphics[width=\linewidth]{Figures/NEATImageDiff1.pdf}
%\endminipage\hfill
%\minipage{0.33\textwidth}
%  \includegraphics[width=\linewidth]{Figures/NEATImageDiff2.pdf}
%\endminipage\hfill
%\minipage{0.33\textwidth}
%  \includegraphics[width=\linewidth]{Figures/NEATImageDiff3.pdf}
%\endminipage
%\caption{Image Differencing results for the CY46 Triplet. (Artifacts such as crater-like formations are seen in the difference images above. This is the result of some celestial bodies being over-exposed.)}
%\label{fig:NEAT_ImgDiff1}
%\end{figure*}

%\begin{figure*}
%\minipage{0.33\textwidth}
%  \includegraphics[width=\linewidth]{Figures/NEATFilteredCentroids1.pdf}
%\endminipage\hfill
%\minipage{0.33\textwidth}
%  \includegraphics[width=\linewidth]{Figures/NEATFilteredCentroids2.pdf}
%\endminipage\hfill
%\minipage{0.33\textwidth}
%  \includegraphics[width=\linewidth]{Figures/NEATFilteredCentroids3.pdf}
%\endminipage
%\caption{Image Differencing results for the CY46 Triplet. Filtered centroids in each image of the sequence.)}
%\label{fig:NEAT_ImgDiff2}
%\end{figure*}

\begin{figure}[t]
\minipage{0.45\textwidth}
  \includegraphics[width=\linewidth]{Figures/NEATLines_LogicalImg.pdf}
\endminipage\hfill
\caption{Trajectory Detection for the CY46 Triplet. (Asteroid trajectory detected is shown in green. True location is in red. 3 Images of the triplet are super-imposed here after registration and thresholding for ease of visualization.)}
\label{fig:NEAT}
\end{figure}
\subsubsection{CATALINA}


\subsection{performance characterization}
We use the following metrics for algorithm testing and validation.  These are used to characterize the performance of the algorithm for the three successive stages: the initial detection of objects, the detection of moving objects and the detection of trajectories:
1.	Precision: The fraction of retrieved/detected objects (celestial bodies, moving objects, trajectories) that are relevant, i.e. correspond to correct detections.
2.	Recall: The fraction of relevant (true) objects (celestial bodies, moving objects, trajectories) that are actually detected/retrieved.
3.	Receiver Operating Characteristic (ROC): A plot of the probability of detection as a function of the probability of false alarm generated by varying a significant parameter of the algorithm (such as the initial detection threshold).  This will detail the performance of the system as the parameters of the image processing pipeline are varied. 
4.	Area under the ROC curve (AUC) that characterizes the performance of detection vs. false alarm with a single metric.
5.	Localization error: For those correctly detected asteroids, we will characterize the localization error by computing the 3D (angular pointing error).  The simulation and the test image sets have the ground truth location of the asteroid in image coordinates.  This true location is compared with the detected trajectory to determine localization error.

The above measures are computed and aggregated over a corpus of images corresponding to a simulated platform and camera.
\subsection{Footprint characterization}
include here some information to the effect that this algorithm is deployable on a reference platform.
Efforts are currently being conducted to translate this algorithm in ...

\section{CONCLUSION}
We describe a pipeline for on-board NEO detection. Such algorithms are critical to allow the detection of moderate to small asteroids from space-based surveys, as these asteroids are not easily seen in earth-based surveys yet they pose a significant potential for damage if they hit the earth. We characterize the algorithmic performance on simulated space-based as well as real earth-based imagery. The algorithmic pipeline is compatible (with regard to CPU and memory usage) with deployment on spacecraft processors. The performance results and computation requirements are promising with regard to the possible future deployment of this algorithm on operational spacecraft processors such as MCP750. 

% Below is an example of how to insert images. Delete the ``\vspace'' line,
% uncomment the preceding line ``\centerline...'' and replace ``imageX.ps''
% with a suitable PostScript file name.
% -------------------------------------------------------------------------
%\begin{figure}[t]
%
%\begin{minipage}[b]{1.0\linewidth}
%  \centering
%  \centerline{\includegraphics[width=8.5cm]{Figures/image1}}
%%  \vspace{2.0cm}
%  \centerline{(a) Result 1}\medskip
%\end{minipage}
%%
%\begin{minipage}[b]{.48\linewidth}
%  \centering
%  \centerline{\includegraphics[width=4.0cm]{Figures/image3}}
%%  \vspace{1.5cm}
%  \centerline{(b) Results 3}\medskip
%\end{minipage}
%\hfill
%\begin{minipage}[b]{0.48\linewidth}
%  \centering
%  \centerline{\includegraphics[width=4.0cm]{Figures/image4}}
%%  \vspace{1.5cm}
%  \centerline{(c) Result 4}\medskip
%\end{minipage}
%%
%\caption{some placeholder figure.}
%\label{fig:res}
%%
%\end{figure}


% To start a new column (but not a new page) and help balance the last-page
% column length use \vfill\pagebreak.
% -------------------------------------------------------------------------
%\vfill
%\pagebreak


% References should be produced using the bibtex program from suitable
% BiBTeX files (here: refs). The IEEEbib.bst bibliography
% style file from IEEE produces unsorted bibliography list.
% -------------------------------------------------------------------------
\bibliographystyle{IEEEbib}
\bibliography{asteroid}

\end{document}
