% Template for ICIP-2015 paper; to be used with:
%          spconf.sty  - ICASSP/ICIP LaTeX style file, and
%          IEEEbib.bst - IEEE bibliography style file.
% --------------------------------------------------------------------------
\documentclass{article}
\usepackage{spconf,amsmath,graphicx}

% Example definitions.
% --------------------
\def\x{{\mathbf x}}
\def\L{{\cal L}}

% Title.
% ------
\title{NEAR EARTH BODY DETECTION and linking}
%
% Single address.
% ---------------
\name{ us the authors us the authors us the authors 
\thanks{This work is supported by NASA Grant XYZ.}}
\address{The Johns Hopkins University  \\ $^{1}$Dept. of Computer Science, $^{2}$Dept. of Applied Mathematics and Statistics, $^{3}$Applied Physics Laboratory}
%
% For example:
% ------------
%\address{School\\
%	Department\\
%	Address}
%
% Two addresses (uncomment and modify for two-address case).
% ----------------------------------------------------------
%\twoauthors
%  {A. Author-one, B. Author-two\sthanks{Thanks to XYZ agency for funding.}}
%	{School A-B\\
%	Department A-B\\
%	Address A-B}
%  {C. Author-three, D. Author-four\sthanks{The fourth author performed the work
%	while at ...}}
%	{School C-D\\
%	Department C-D\\
%	Address C-D}
%
\begin{document}
%\ninept
%
\maketitle
%
\begin{abstract}
While most asteroid population discovery to-date has been accomplished  by earth-based telescopes, it is speculated that most of the smaller Near Earth Objects (NEOs), up to 140 meters in diameter, whose impact can create substantial city-size damage, have not yet been discovered.  Further, there are asteroids that cannot be detected with an earth-based telescope given their size or the location of the Sun.  Our objective is therefore to develop an efficient asteroid detection and identification algorithm that can be hosted on-board a spacecraft.  By having on-board algorithms, the system would also minimize the need to downlink entire images taken by a space-based telescope. We describe one such image processing pipeline we have developed for asteroid detection and also characterize its performance.
 
\end{abstract}
%
\begin{keywords}
Asteroid detection and identification, on-board autonomous algorithms 
\end{keywords}
%
\section{Introduction}
\label{sec:intro}

NASA has a congressional mandate to discover all Near-Earth Objects (NEOs) at least 1 kilometer in diameter.  Fortunately, 95\% of the NEOs larger than 1 km that have been discovered are likely not to impact Earth.  
Near-Earth Object search programs~\cite{stokes2002near} are currently almost exclusively accomplished by earth-based telescopes such as MIT's LINEAR \cite{evans2003detection}  project, the NEAT~\cite{neat2014} program, the Catalina Sky Survey, or  Pan-STARRS~\cite{denneau2013pan}.  (An exception is JPL's spacecraft-based WISE telescope brought out of hibernation to characterize NEOs in the 3.4 and 4.6 micron infrared bands~\cite{wise2014}). 

It is notable however that the NEO that has impacted Chelyabinsk, Russia on 15 February 2013 was only about 17 meters in size.  The impact of a 50 meter asteroid that caused the Tunguska Event of 1908 could have destroyed an entire city or metropolitan area. It is estimated that only a relatively small fraction of those so called ``city-killing'' asteroids, particularly objects less than 140 meters in diameter, have been discovered to date. Because of their size, atmospheric effects and the location of the sun, some of these NEOs cannot easily be detected with an earth-based telescope.  

Our focus here is therefore on developing an algorithm that can be hosted on-board a spacecraft.  Understanding that there are processing and resources (memory) constraints on-board, our algorithm design needs to not only meet the performance objectives of detecting and identifying asteroids using a space-based telescope, but it also needs to have a small footprint to be implementable with the limited on-board resources.  This paper describes an agile algorithm candidate that is being investigated as well as our use of representative space-based imagery for testing it.

There is prior image analysis work for asteroid detection for ground-based image observations. In~\cite{denneau2013pan}, a reference processing system for detection and identification of asteroid is reported, named the Pan-STARRS Moving Object Processing System (MOPS). This pipeline aims at identifying moving objects in our solar system and linking those detections within and between night observations. It attributes those detections to known objects, calculates initial and differentially corrected orbits for linked detections, recovering detections when they exist, and orbit identification. Most proposed pipelines for earth based detection include a step to combine together images to create a high S/N static-sky image that is subtracted from the current master image to obtain a difference image containing only transient sources. Examples include \cite{shao2014finding}, where a shift-and-add technique is used to improve  signal to noise ratio and then synthetically creating long exposure images to facilitate the detection of trajectories. A related shift-and-add method using a median  image rather than an average image is reported in~\cite{yanagisawa2005automatic}. A match filter is used for asteroid detection and matching in \cite{gural2005matched}. In \cite{kubica2005variable,kubica2005multiple,kubica2007efficient}, tree based searches (including KD-trees) are used for efficient linking of successive asteroids detections and finding sets of observation points that can be fitted with an inherent motion model, through an exhaustive search for all possible linkages that satisfy the expected model constraints.
	
In Section~\ref{sec:approach} we describe an agile pipeline that can be deployed on an on board system. This approach is both small-footprinted and efficient, and uses Principal Component Analysis rather than tree-based searches to aggregate object detections into trajectories. Since on-board surveys are not available we describe in Section~\ref{sec:experiments} how we validate and test our method using simulated images as well as earth based and space-based image datasets that contain associated ground truth.

\section{Approach}
\label{sec:approach}

We describe the main components of our image processing pipeline (shown also in the flow chart in figure\ref{}) addressing detection and linking of asteroids seen in images acquired from space-based platforms.  At a high level, the pipeline is summarized as follows: as input to the pipeline is a sequence of time-lapse images. Pre-processing and image registration are used to bring the images into alignment with a common image of reference.  Detection of bright bodies is done via thresholding, followed by logical differencing between each image and a reference image containing objects that are common to all registered images in the sequence of images.  After differencing, which allows us to detect moving objects, we use an additional step to filter detections based on size and shape considerations (using tools such as connected components and morphological filters).  A list of detection coordinates is then generated.  Candidate trajectories are then generated for these coordinates by checking rectilinearity (via Principal Component Analysis).  A last verification step further checks additional conditions such as the fact that candidate trajectories must be composed of temporally consecutive mover observations.  Each module that is detailed next. 

{\bf Image Pre-Processing}
We standardize the input images by using a series of photometric and geometric transformations.  This includes employing techniques to reduce noise (e.g. median filter) and artifacts that are consequences of the intrinsic properties of the acquisition device.   This coordinate information can then be stored in the FITS (Flexible Image Transport System) image header for later use in the Trajectory Verification stage.

{\bf Image Registration}
Image registration refers to finding the transformation that would align multiple images of the same scene that are obtained at different times from slightly different viewpoints.  We aim to align each image in the sequence to a common reference image so that the stars in the background line up in all images.  In the case of a triplet of images, typically the second image is used as the reference image.  We estimate and then apply the necessary translation, rotation, scaling (similarity transformation) and/or skew (full affine transformation) to all images in the sequence such that image objects (stars) are mapped into the same pixels location and so that all transformed images have the same spatial resolution.  Image Registration using mutual information~\cite{viola1997alignment} and cross-correlation as similarity measures are used. 

{\bf Image Logical Differencing}
A global thresholding is applied to the registered image for detecting asteroids and suppressing background noise.  As asteroids are typically very faint compared to the surrounding stars, the selection of the detection threshold impacts false alarm rate.  Thresholding yields binary detection images.  The set of all binary images is then used to generate an intersection image that contains objects that occur in all images in the sequence.  This is followed by logical differencing whereby we produce a set of difference images by intersecting the corresponding binary image with the negative of the common intersection image.  This operation provides a list of candidate detections for each image in the sequence.  While the logical differencing results in good detections, additional artifacts such as crater-like formations are seen as a result of some of the celestial bodies being over-exposed.  To mitigate this artifact, we subsequently perform filtering out of hollow objects as well as filtering based on object size.

{\bf Trajectory Detection}
The list of centroids of moving objects obtained from image differencing defines a set of candidate rectilinear trajectories.  The goal is to find a subset of centroids that fit a linear model. Even though the model is simple, the set of filtered centroids potentially has a high number of noisy points (falsely detected movers), and the cardinality of the set of all candidate trajectories increases exponentially with the number of detections, thus requiring subsequent pruning of this set.  We search through all possible trajectories via Principal Component Analysis (PCA) and compute the ratio of eigenvalues, $\lambda_{1}/(\lambda_{1}+ \lambda_{2})$ in order to develop a line confidence score for each candidate trajectory, and choose candidate lines for which this ratio exceeds a threshold. Using the candidate trajectories thus found, we then enforce temporal order of detection by using the sign of the projection on the principal eigenvector.  As the final step, we further eliminate false positives by ensuring that the distance between projections is proportional to the time interval between images.  

\section{Experiments}
\label{sec:experiments}

\subsection{Simulated Imagery}
\label{ssec:simulated}

\subsection{Real Imagery}
\label{ssec:real}

\subsubsection{NEAT}
\subsubsection{CATALINA}


\subsection{performance characterization}

\section{CONCLUSION}

% Below is an example of how to insert images. Delete the ``\vspace'' line,
% uncomment the preceding line ``\centerline...'' and replace ``imageX.ps''
% with a suitable PostScript file name.
% -------------------------------------------------------------------------
%\begin{figure}[t]
%
%\begin{minipage}[b]{1.0\linewidth}
%  \centering
%  \centerline{\includegraphics[width=8.5cm]{Figures/image1}}
%%  \vspace{2.0cm}
%  \centerline{(a) Result 1}\medskip
%\end{minipage}
%%
%\begin{minipage}[b]{.48\linewidth}
%  \centering
%  \centerline{\includegraphics[width=4.0cm]{Figures/image3}}
%%  \vspace{1.5cm}
%  \centerline{(b) Results 3}\medskip
%\end{minipage}
%\hfill
%\begin{minipage}[b]{0.48\linewidth}
%  \centering
%  \centerline{\includegraphics[width=4.0cm]{Figures/image4}}
%%  \vspace{1.5cm}
%  \centerline{(c) Result 4}\medskip
%\end{minipage}
%%
%\caption{some placeholder figure.}
%\label{fig:res}
%%
%\end{figure}


% To start a new column (but not a new page) and help balance the last-page
% column length use \vfill\pagebreak.
% -------------------------------------------------------------------------
%\vfill
%\pagebreak


% References should be produced using the bibtex program from suitable
% BiBTeX files (here: refs). The IEEEbib.bst bibliography
% style file from IEEE produces unsorted bibliography list.
% -------------------------------------------------------------------------
\bibliographystyle{IEEEbib}
\bibliography{asteroid}

\end{document}
